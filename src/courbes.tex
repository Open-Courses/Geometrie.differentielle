\chapter{Courbes}

\section{Courbes, courbes paramétrées (par longueur d'arc), vitesses}

\begin{definition}
	\textbf{Une courbe $\alpha$} est une fonction
	$\mathcal{C}^{\infty}$ d'un intervalle $I \subseteq \real$ dans $\real^{n}$.
\end{definition}

\begin{remarque}
	Nous pourrions nous restreindre aux courbes de classe $\mathcal{C}^{n}$ dans
	la plupart des cas. Pour éviter les discussions, nous travaillerons qu'avec
	des courbes indéfiniment différentiables.
\end{remarque}

\begin{exemple}
	
\end{exemple}

\begin{definition}
	On définit \textbf{la vitesse de la courbe} $\alpha$ comme la dérivée de la
	courbe $\alpha$. On la note $\dot{\alpha}$ ou $t : I \rightarrow \real^{n} :
	s \rightarrow t(s) = \dot{\alpha}(s)$.
\end{definition}

\begin{definition}
	Soit $t \in I$ et une courbe $\alpha : I \rightarrow \real^{n}$. On dit que
	\textbf{$t$ est un point régulier} si $\dot{\alpha}(t) \neq 0_{\real^{n}}$.
	Sinon, $t$ est dit \textbf{singulier}.
\end{definition}

\begin{definition}
	Une courbe est \textbf{paramètrée} si elle ne possède aucun point singulier.
\end{definition}

\begin{exemple}
	
\end{exemple}

\begin{definition}
	Une courbe paramétrée est dite \textbf{paramétrée par longeur d'arc} si elle
	est sa vitesse est de norme $1$ en chaque point, c'est-à-dire $\forall s \in
	I$, $\norm{\dot{\alpha}(s)} = 1$.
\end{definition}

\section{Courbure, torsion}

\begin{definition}
	Soit $\alpha$ une courbe paramètrée par longueur d'arc. On définit
	\textbf{la courbure de $\alpha$} comme sa dérivée seconde. La courbure de
	$\alpha$ en un point $s \in I$ vaut $\ddot{\alpha}(s)$.
\end{definition}
